% Options for packages loaded elsewhere
\PassOptionsToPackage{unicode}{hyperref}
\PassOptionsToPackage{hyphens}{url}
\PassOptionsToPackage{dvipsnames,svgnames,x11names}{xcolor}
%
\documentclass[
  letterpaper,
  DIV=11,
  numbers=noendperiod]{scrartcl}

\usepackage{amsmath,amssymb}
\usepackage{iftex}
\ifPDFTeX
  \usepackage[T1]{fontenc}
  \usepackage[utf8]{inputenc}
  \usepackage{textcomp} % provide euro and other symbols
\else % if luatex or xetex
  \usepackage{unicode-math}
  \defaultfontfeatures{Scale=MatchLowercase}
  \defaultfontfeatures[\rmfamily]{Ligatures=TeX,Scale=1}
\fi
\usepackage{lmodern}
\ifPDFTeX\else  
    % xetex/luatex font selection
\fi
% Use upquote if available, for straight quotes in verbatim environments
\IfFileExists{upquote.sty}{\usepackage{upquote}}{}
\IfFileExists{microtype.sty}{% use microtype if available
  \usepackage[]{microtype}
  \UseMicrotypeSet[protrusion]{basicmath} % disable protrusion for tt fonts
}{}
\makeatletter
\@ifundefined{KOMAClassName}{% if non-KOMA class
  \IfFileExists{parskip.sty}{%
    \usepackage{parskip}
  }{% else
    \setlength{\parindent}{0pt}
    \setlength{\parskip}{6pt plus 2pt minus 1pt}}
}{% if KOMA class
  \KOMAoptions{parskip=half}}
\makeatother
\usepackage{xcolor}
\setlength{\emergencystretch}{3em} % prevent overfull lines
\setcounter{secnumdepth}{-\maxdimen} % remove section numbering
% Make \paragraph and \subparagraph free-standing
\ifx\paragraph\undefined\else
  \let\oldparagraph\paragraph
  \renewcommand{\paragraph}[1]{\oldparagraph{#1}\mbox{}}
\fi
\ifx\subparagraph\undefined\else
  \let\oldsubparagraph\subparagraph
  \renewcommand{\subparagraph}[1]{\oldsubparagraph{#1}\mbox{}}
\fi

\usepackage{color}
\usepackage{fancyvrb}
\newcommand{\VerbBar}{|}
\newcommand{\VERB}{\Verb[commandchars=\\\{\}]}
\DefineVerbatimEnvironment{Highlighting}{Verbatim}{commandchars=\\\{\}}
% Add ',fontsize=\small' for more characters per line
\usepackage{framed}
\definecolor{shadecolor}{RGB}{241,243,245}
\newenvironment{Shaded}{\begin{snugshade}}{\end{snugshade}}
\newcommand{\AlertTok}[1]{\textcolor[rgb]{0.68,0.00,0.00}{#1}}
\newcommand{\AnnotationTok}[1]{\textcolor[rgb]{0.37,0.37,0.37}{#1}}
\newcommand{\AttributeTok}[1]{\textcolor[rgb]{0.40,0.45,0.13}{#1}}
\newcommand{\BaseNTok}[1]{\textcolor[rgb]{0.68,0.00,0.00}{#1}}
\newcommand{\BuiltInTok}[1]{\textcolor[rgb]{0.00,0.23,0.31}{#1}}
\newcommand{\CharTok}[1]{\textcolor[rgb]{0.13,0.47,0.30}{#1}}
\newcommand{\CommentTok}[1]{\textcolor[rgb]{0.37,0.37,0.37}{#1}}
\newcommand{\CommentVarTok}[1]{\textcolor[rgb]{0.37,0.37,0.37}{\textit{#1}}}
\newcommand{\ConstantTok}[1]{\textcolor[rgb]{0.56,0.35,0.01}{#1}}
\newcommand{\ControlFlowTok}[1]{\textcolor[rgb]{0.00,0.23,0.31}{#1}}
\newcommand{\DataTypeTok}[1]{\textcolor[rgb]{0.68,0.00,0.00}{#1}}
\newcommand{\DecValTok}[1]{\textcolor[rgb]{0.68,0.00,0.00}{#1}}
\newcommand{\DocumentationTok}[1]{\textcolor[rgb]{0.37,0.37,0.37}{\textit{#1}}}
\newcommand{\ErrorTok}[1]{\textcolor[rgb]{0.68,0.00,0.00}{#1}}
\newcommand{\ExtensionTok}[1]{\textcolor[rgb]{0.00,0.23,0.31}{#1}}
\newcommand{\FloatTok}[1]{\textcolor[rgb]{0.68,0.00,0.00}{#1}}
\newcommand{\FunctionTok}[1]{\textcolor[rgb]{0.28,0.35,0.67}{#1}}
\newcommand{\ImportTok}[1]{\textcolor[rgb]{0.00,0.46,0.62}{#1}}
\newcommand{\InformationTok}[1]{\textcolor[rgb]{0.37,0.37,0.37}{#1}}
\newcommand{\KeywordTok}[1]{\textcolor[rgb]{0.00,0.23,0.31}{#1}}
\newcommand{\NormalTok}[1]{\textcolor[rgb]{0.00,0.23,0.31}{#1}}
\newcommand{\OperatorTok}[1]{\textcolor[rgb]{0.37,0.37,0.37}{#1}}
\newcommand{\OtherTok}[1]{\textcolor[rgb]{0.00,0.23,0.31}{#1}}
\newcommand{\PreprocessorTok}[1]{\textcolor[rgb]{0.68,0.00,0.00}{#1}}
\newcommand{\RegionMarkerTok}[1]{\textcolor[rgb]{0.00,0.23,0.31}{#1}}
\newcommand{\SpecialCharTok}[1]{\textcolor[rgb]{0.37,0.37,0.37}{#1}}
\newcommand{\SpecialStringTok}[1]{\textcolor[rgb]{0.13,0.47,0.30}{#1}}
\newcommand{\StringTok}[1]{\textcolor[rgb]{0.13,0.47,0.30}{#1}}
\newcommand{\VariableTok}[1]{\textcolor[rgb]{0.07,0.07,0.07}{#1}}
\newcommand{\VerbatimStringTok}[1]{\textcolor[rgb]{0.13,0.47,0.30}{#1}}
\newcommand{\WarningTok}[1]{\textcolor[rgb]{0.37,0.37,0.37}{\textit{#1}}}

\providecommand{\tightlist}{%
  \setlength{\itemsep}{0pt}\setlength{\parskip}{0pt}}\usepackage{longtable,booktabs,array}
\usepackage{calc} % for calculating minipage widths
% Correct order of tables after \paragraph or \subparagraph
\usepackage{etoolbox}
\makeatletter
\patchcmd\longtable{\par}{\if@noskipsec\mbox{}\fi\par}{}{}
\makeatother
% Allow footnotes in longtable head/foot
\IfFileExists{footnotehyper.sty}{\usepackage{footnotehyper}}{\usepackage{footnote}}
\makesavenoteenv{longtable}
\usepackage{graphicx}
\makeatletter
\def\maxwidth{\ifdim\Gin@nat@width>\linewidth\linewidth\else\Gin@nat@width\fi}
\def\maxheight{\ifdim\Gin@nat@height>\textheight\textheight\else\Gin@nat@height\fi}
\makeatother
% Scale images if necessary, so that they will not overflow the page
% margins by default, and it is still possible to overwrite the defaults
% using explicit options in \includegraphics[width, height, ...]{}
\setkeys{Gin}{width=\maxwidth,height=\maxheight,keepaspectratio}
% Set default figure placement to htbp
\makeatletter
\def\fps@figure{htbp}
\makeatother

\KOMAoption{captions}{tableheading}
\makeatletter
\@ifpackageloaded{caption}{}{\usepackage{caption}}
\AtBeginDocument{%
\ifdefined\contentsname
  \renewcommand*\contentsname{Índice}
\else
  \newcommand\contentsname{Índice}
\fi
\ifdefined\listfigurename
  \renewcommand*\listfigurename{Lista de Figuras}
\else
  \newcommand\listfigurename{Lista de Figuras}
\fi
\ifdefined\listtablename
  \renewcommand*\listtablename{Lista de Tabelas}
\else
  \newcommand\listtablename{Lista de Tabelas}
\fi
\ifdefined\figurename
  \renewcommand*\figurename{Figura}
\else
  \newcommand\figurename{Figura}
\fi
\ifdefined\tablename
  \renewcommand*\tablename{Tabela}
\else
  \newcommand\tablename{Tabela}
\fi
}
\@ifpackageloaded{float}{}{\usepackage{float}}
\floatstyle{ruled}
\@ifundefined{c@chapter}{\newfloat{codelisting}{h}{lop}}{\newfloat{codelisting}{h}{lop}[chapter]}
\floatname{codelisting}{Listagem}
\newcommand*\listoflistings{\listof{codelisting}{Lista de Listagens}}
\makeatother
\makeatletter
\makeatother
\makeatletter
\@ifpackageloaded{caption}{}{\usepackage{caption}}
\@ifpackageloaded{subcaption}{}{\usepackage{subcaption}}
\makeatother
\ifLuaTeX
\usepackage[bidi=basic]{babel}
\else
\usepackage[bidi=default]{babel}
\fi
\babelprovide[main,import]{portuguese}
% get rid of language-specific shorthands (see #6817):
\let\LanguageShortHands\languageshorthands
\def\languageshorthands#1{}
\ifLuaTeX
  \usepackage{selnolig}  % disable illegal ligatures
\fi
\usepackage{bookmark}

\IfFileExists{xurl.sty}{\usepackage{xurl}}{} % add URL line breaks if available
\urlstyle{same} % disable monospaced font for URLs
\hypersetup{
  pdftitle={Lotações Preponderantes},
  pdfauthor={Ênio Lopes},
  pdflang={pt},
  colorlinks=true,
  linkcolor={blue},
  filecolor={Maroon},
  citecolor={Blue},
  urlcolor={Blue},
  pdfcreator={LaTeX via pandoc}}

\title{Lotações Preponderantes}
\author{Ênio Lopes}
\date{01/04/2026}

\begin{document}
\maketitle

\subsection{Lotações
Preponderantes}\label{lotauxe7uxf5es-preponderantes}

Primeiramente deve-se gerar relatório de designações do período de
interesse no Sistema de Gratificação de Magistrados (SGM).

Em Jan/2025 a funcionalidade de exportar as designações para \emph{.xls}
não estava funcionando no SGM. Assim, tive que transformar o arquivo
\emph{.pdf} para \emph{.csv} a fim de realizar o cálculo das lotações
preponderantes dos magistrados.

\begin{Shaded}
\begin{Highlighting}[]
\CommentTok{\#Pacotes necessários}
\FunctionTok{library}\NormalTok{(dplyr)}
\FunctionTok{library}\NormalTok{(lubridate)}
\FunctionTok{library}\NormalTok{(stringr)}
\FunctionTok{library}\NormalTok{(janitor)}

\NormalTok{designacoes}\OtherTok{\textless{}{-}}\FunctionTok{read.csv}\NormalTok{(}\StringTok{"RelatorioDesignacoes\_2024.csv"}\NormalTok{,}\AttributeTok{h=}\NormalTok{T,}\AttributeTok{encoding =} \StringTok{"UTF{-}8"}\NormalTok{)}
\NormalTok{designacoes }\OtherTok{\textless{}{-}}\NormalTok{ designacoes }\SpecialCharTok{\%\textgreater{}\%} 
  \FunctionTok{select}\NormalTok{(}\SpecialCharTok{{-}}\NormalTok{X)}
\NormalTok{designacoes }\OtherTok{\textless{}{-}}\NormalTok{ designacoes }\SpecialCharTok{\%\textgreater{}\%} 
  \FunctionTok{clean\_names}\NormalTok{()}
\NormalTok{designacoes}\SpecialCharTok{$}\NormalTok{de }\OtherTok{\textless{}{-}} \FunctionTok{dmy}\NormalTok{(designacoes}\SpecialCharTok{$}\NormalTok{de)}
\NormalTok{designacoes}\SpecialCharTok{$}\NormalTok{ate }\OtherTok{\textless{}{-}} \FunctionTok{dmy}\NormalTok{(designacoes}\SpecialCharTok{$}\NormalTok{ate)}

\NormalTok{designacoes}\SpecialCharTok{$}\NormalTok{unidade }\OtherTok{\textless{}{-}} \FunctionTok{str\_replace}\NormalTok{(designacoes}\SpecialCharTok{$}\NormalTok{unidade,}\StringTok{"a"}\NormalTok{,}\StringTok{"ª"}\NormalTok{)}

\NormalTok{unidades}\OtherTok{\textless{}{-}}\FunctionTok{read.csv2}\NormalTok{(}\StringTok{"Unidades.csv"}\NormalTok{,}\AttributeTok{h=}\NormalTok{T)}

\NormalTok{designacoes}\OtherTok{\textless{}{-}}\NormalTok{designacoes}\SpecialCharTok{\%\textgreater{}\%}
  \FunctionTok{mutate}\NormalTok{(}\AttributeTok{FIM=}\FunctionTok{ifelse}\NormalTok{(ate}\SpecialCharTok{\textgreater{}}\FunctionTok{ymd}\NormalTok{(}\StringTok{"2024{-}12{-}31"}\NormalTok{),}\FunctionTok{ymd}\NormalTok{(}\StringTok{"2024{-}12{-}31"}\NormalTok{),ate))}\SpecialCharTok{\%\textgreater{}\%}
  \FunctionTok{mutate}\NormalTok{(}\AttributeTok{INICIO=}\FunctionTok{ifelse}\NormalTok{(de}\SpecialCharTok{\textless{}}\FunctionTok{ymd}\NormalTok{(}\StringTok{"2024{-}01{-}01"}\NormalTok{),}\FunctionTok{ymd}\NormalTok{(}\StringTok{"2024{-}01{-}01"}\NormalTok{),de))}
\NormalTok{designacoes}\SpecialCharTok{$}\NormalTok{FIM}\OtherTok{\textless{}{-}}\FunctionTok{as.Date}\NormalTok{(designacoes}\SpecialCharTok{$}\NormalTok{FIM,}\AttributeTok{origin =} \StringTok{"1970{-}01{-}01"}\NormalTok{)}
\NormalTok{designacoes}\SpecialCharTok{$}\NormalTok{INICIO}\OtherTok{\textless{}{-}}\FunctionTok{as.Date}\NormalTok{(designacoes}\SpecialCharTok{$}\NormalTok{INICIO,}\AttributeTok{origin =} \StringTok{"1970{-}01{-}01"}\NormalTok{)}
\NormalTok{designacoes}\SpecialCharTok{$}\NormalTok{DIAS}\OtherTok{\textless{}{-}}\NormalTok{designacoes}\SpecialCharTok{$}\NormalTok{FIM}\SpecialCharTok{{-}}\NormalTok{designacoes}\SpecialCharTok{$}\NormalTok{INICIO}\SpecialCharTok{+}\DecValTok{1}

\NormalTok{desig\_juiz}\OtherTok{\textless{}{-}}\NormalTok{designacoes}\SpecialCharTok{\%\textgreater{}\%}
  \FunctionTok{group\_by}\NormalTok{(magistrado,unidade)}\SpecialCharTok{\%\textgreater{}\%}
  \FunctionTok{summarise}\NormalTok{(}\AttributeTok{max\_total\_dias\_vt=}\FunctionTok{as.numeric}\NormalTok{(}\FunctionTok{sum}\NormalTok{(DIAS)))}

\NormalTok{juizes}\OtherTok{\textless{}{-}}\FunctionTok{unique}\NormalTok{(desig\_juiz}\SpecialCharTok{$}\NormalTok{magistrado)}
\NormalTok{resumo}\OtherTok{\textless{}{-}}\FunctionTok{data.frame}\NormalTok{(}\AttributeTok{magistrado=}\FunctionTok{character}\NormalTok{(}\DecValTok{0}\NormalTok{),}\AttributeTok{unidade=}\FunctionTok{character}\NormalTok{(}\DecValTok{0}\NormalTok{),}\AttributeTok{max\_total\_dias\_vt=}\FunctionTok{numeric}\NormalTok{(}\DecValTok{0}\NormalTok{))}

\ControlFlowTok{for}\NormalTok{ (i }\ControlFlowTok{in}\NormalTok{ juizes)\{}
\NormalTok{  filtro}\OtherTok{\textless{}{-}}\NormalTok{desig\_juiz}\SpecialCharTok{\%\textgreater{}\%}
    \FunctionTok{filter}\NormalTok{(magistrado}\SpecialCharTok{==}\NormalTok{i)}
\NormalTok{  agg}\OtherTok{\textless{}{-}}\NormalTok{filtro}\SpecialCharTok{\%\textgreater{}\%}
    \FunctionTok{group\_by}\NormalTok{(magistrado)}\SpecialCharTok{\%\textgreater{}\%}
    \FunctionTok{summarise}\NormalTok{(}\AttributeTok{tot\_dias=}\FunctionTok{sum}\NormalTok{(max\_total\_dias\_vt))}
\NormalTok{  filtro}\OtherTok{\textless{}{-}}\NormalTok{filtro}\SpecialCharTok{\%\textgreater{}\%}
    \FunctionTok{filter}\NormalTok{(max\_total\_dias\_vt}\SpecialCharTok{==}\FunctionTok{max}\NormalTok{(max\_total\_dias\_vt))}
\NormalTok{  filtro}\OtherTok{\textless{}{-}}\FunctionTok{left\_join}\NormalTok{(filtro,agg,}\AttributeTok{by=}\StringTok{"magistrado"}\NormalTok{)}
\NormalTok{  resumo}\OtherTok{\textless{}{-}}\FunctionTok{rbind}\NormalTok{(resumo,filtro)}
\NormalTok{\}}

\NormalTok{resumo}\OtherTok{\textless{}{-}}\FunctionTok{left\_join}\NormalTok{(resumo,unidades,}\AttributeTok{by=}\FunctionTok{c}\NormalTok{(}\StringTok{"unidade"}\OtherTok{=}\StringTok{"UNIDADE"}\NormalTok{))}\SpecialCharTok{\%\textgreater{}\%}
  \FunctionTok{filter}\NormalTok{(TIPO}\SpecialCharTok{==}\StringTok{"VT"}\NormalTok{)}
\end{Highlighting}
\end{Shaded}




\end{document}
